\documentclass[12pt]{article}
\usepackage{graphicx}
\usepackage{geometry}
\usepackage[utf8]{inputenc}
\usepackage[T1]{fontenc}
\usepackage{ulem}
\usepackage{enumitem}

\geometry{margin=1in}

\begin{document}

\thispagestyle{empty}

% --- Trois colonnes symétriques ---
\begin{minipage}{0.3\linewidth}
    \begin{center}
        \textbf{RÉPUBLIQUE DU CAMEROUN}\\
        *****\\
        Paix - Travail - Patrie\\
        *****\\[0.3cm]

        \textbf{UNIVERSITÉ DE YAOUNDÉ I}\\
        *****\\[0.3cm]

        ÉCOLE NATIONALE SUPÉRIEURE\\
        POLYTECHNIQUE DE YAOUNDÉ\\
        *****\\[0.3cm]

        \textbf{DÉPARTEMENT DE GÉNIE}\\
        ***** 
    \end{center}
\end{minipage}
\begin{minipage}{0.35\linewidth}
    \begin{center}
        % Logos au centre
        \includegraphics[width=3cm]{uy1.png}\\[0.5cm]
        \includegraphics[width=3cm]{polytech.png}
    \end{center}
\end{minipage}
\begin{minipage}{0.3\linewidth}
    \begin{center}
        \textbf{REPUBLIC OF CAMEROON}\\
        *****\\
        Peace - Work - Fatherland\\
        *****\\[0.3cm]

        \textbf{UNIVERSITY OF YAOUNDE I}\\
        *****\\[0.3cm]

        NATIONAL ADVANCED SCHOOL\\
        OF ENGINEERING OF YAOUNDE\\
        *****\\[0.3cm]

        \textbf{COMPUTER ENGINEERING}\\
        ***** 
    \end{center}
\end{minipage}

\vspace{2cm}

% --- Sujet au centre ---
\begin{center}
    {\LARGE \textbf{Techniques et pratiques de l'investigation numérique}}\\[0.5cm]
    {\large \textbf{Sous le thème :}Resumé Techniques et pratiques de l'investigation numérique}\\[2cm]

    \textbf{{EMBOLO MVOGO SHAWN DOUGLAS}}\\[0.8cm]
    \textbf{{22P072}}\\[0.8cm]
    \textbf{{Filiére : CIN4}}\\[0.8cm]
    Sous la supervision de : \textbf{Mr Minka}\\[1.5cm]

    Année Scolaire : \textbf{2025 -- 2026}
\end{center}

\section*{Résumé de l’Investigation Numérique Éthique}

L’investigation numérique éthique constitue un pilier central de la cybersécurité et de la lutte contre la criminalité informatique. Elle dépasse le simple cadre technique pour questionner les fondements de la vérité, de la confiance et de la justice à l’ère numérique. Elle exige un équilibre subtil entre compétences techniques pointues, sagesse, humilité et rigueur morale. 

Le \textbf{Contrat Déontologique de l’Investigateur Numérique} représente un engagement moral entre l’apprenant et la communauté des professionnels, soulignant que la maîtrise des outils et des techniques numériques confère un pouvoir considérable et que l’investigation numérique n’est jamais neutre. Chaque technique maîtrisée ou outil utilisé confère une influence sur les systèmes numériques et sur les vies qui y sont connectées.

\section{Engagements et Serment de l’Investigateur}
Le serment de l’investigateur numérique implique des engagements fondamentaux :

\begin{itemize}[leftmargin=*]
    \item Utiliser ses compétences exclusivement à des fins légitimes, autorisées et éthiques.
    \item Respecter strictement les cadres juridiques nationaux et internationaux.
    \item Préserver l’intégrité des systèmes et des données analysées.
    \item Protéger la confidentialité des informations.
    \item Assurer une traçabilité complète et irréprochable de toutes les actions.
\end{itemize}

\subsection{Limitations éthiques}
L’investigateur s’engage à ne jamais utiliser ses compétences pour :
\begin{itemize}[leftmargin=*]
    \item Porter atteinte à la vie privée sans mandat légitime.
    \item Compromettre l’intégrité des systèmes sans autorisation.
    \item Altérer ou détruire des preuves numériques.
    \item Faciliter des activités illicites ou malveillantes.
\end{itemize}

\subsection{Les quatre piliers fondamentaux}
\begin{itemize}[leftmargin=*]
    \item \textbf{Intégrité} : véracité des conclusions, transparence des méthodes et reconnaissance des limites.
    \item \textbf{Proportionalité} : adéquation des moyens aux fins, minimisation de l’intrusion et respect de la vie privée.
    \item \textbf{Responsabilité} : acceptation des conséquences de ses actions, devoir de vigilance et engagement dans la formation continue.
    \item \textbf{Service} : mise des compétences au service de la justice, de la vérité et de la protection des droits.
\end{itemize}

\subsection{Les dix commandements de l’investigateur}
\begin{enumerate}[leftmargin=*]
    \item Ne pas causer de dommages aux systèmes investigués.
    \item Respecter la vie privée et la dignité des personnes.
    \item Maintenir une chaîne de custody irréprochable.
    \item Documenter intégralement les processus et décisions.
    \item Reconnaître ses limites.
    \item Résister aux pressions contraires à l’éthique.
    \item Protéger les données sensibles en sa garde.
    \item Témoigner avec honnêteté et objectivité.
    \item Contribuer au développement de la discipline.
    \item Honorer la confiance que la société place en soi.
\end{enumerate}

\section{Philosophie et Fondements}
L’investigation numérique explore ses dimensions philosophiques, épistémologiques, éthiques et ontologiques. Elle interroge la nature de l’existence numérique et les relations entre l’être physique et son double numérique. La société numérique est confrontée au paradoxe entre transparence et droit à l’intimité, et l’investigateur opère à cette intersection délicate.

\subsection{Épistémologie de la Preuve Numérique}
La preuve numérique se distingue de la preuve traditionnelle par son immatérialité, sa volatilité, sa mutabilité et son authenticité dépendante de la chaîne de confiance. Elle ajoute une dimension temporelle complexe, où chaque événement laisse une empreinte digitale unique.

\section{Histoire et Évolution}
\subsection{Prémices (1970-1990)}
\begin{itemize}[leftmargin=*]
    \item 1971 : The Creeper, premier ver informatique.
    \item 1979 : Première saisie de données par le FBI.
    \item 1983 : Affaire des 414s → création du Computer Fraud and Abuse Act (1986).
    \item 1984 : Dan Farmer formalise le concept de Computer Forensics.
\end{itemize}

\subsection{Professionnalisation (1990-2000)}
Structuration de la discipline et adoption d’outils spécialisés pour faire face à l’augmentation des cybermenaces.

\subsection{Standardisation (2000-2010)}
Publication de la RFC 3227 et développement de frameworks comme Sleuth Kit pour uniformiser les pratiques.

\subsection{Big Data et Cloud (2010-2020)}
Adaptation aux volumes massifs de données et au cloud computing, illustrée par les affaires Panama Papers (2016) et WannaCry (2017).

\section{Fondements Théoriques}
\subsection{Principe de Locard Numérique}

Le principe original d'Édmond Locard stipule que ``toute action laisse une trace''. En investigation numérique, ce principe se décline en deux catégories de traces :

\begin{itemize}[leftmargin=*]
    \item \textbf{Traces Primaires} :
    \begin{itemize}[leftmargin=*]
        \item \textit{Logs système} : Enregistrements horodatés des événements.
        \item \textit{Artefacts de registre} : Modifications dans les bases de registre.
        \item \textit{Fichiers temporaires} : Incluant le cache, le swap et les fichiers d'hibernation.
    \end{itemize}
    \item \textbf{Traces Secondaires} :
    \begin{itemize}[leftmargin=*]
        \item \textit{Métadonnées} : Telles que les données EXIF, les timestamps et les propriétés de fichiers.
        \item \textit{Corrélations réseau} : Incluant les flux NetFlow et les captures PCAP.
        \item \textit{Empreintes comportementales} : Des schémas d'utilisation révélant des patterns d'activité.
    \end{itemize}
\end{itemize}

\subsection{Modèles d’Investigation}

Plusieurs modèles structurés guident l'approche des investigations numériques pour assurer une méthodologie rigoureuse :

\begin{itemize}[leftmargin=*]
    \item \textbf{Le Modèle DFRWS (Digital Forensic Research Workshop Framework) (2001)} :
    \begin{enumerate}[leftmargin=*]
        \item \textit{Identification} : Reconnaître les incidents.
        \item \textit{Préservation} : Isoler et protéger les preuves.
        \item \textit{Collection} : Acquérir les preuves de manière méthodique.
        \item \textit{Examination} : Réaliser une analyse détaillée.
        \item \textit{Analysis} : Corréler les informations et reconstituer les événements.
        \item \textit{Presentation} : Préparer le rapport et le témoignage.
    \end{enumerate}

    \item \textbf{Le Modèle de Casey (Enhanced Integrated Digital Investigation Process) (2004)} :
    \begin{enumerate}[leftmargin=*]
        \item Phase 1 : \textit{Readiness} (Préparation)
        \item Phase 2 : \textit{Deployment} (Déploiement)
        \item Phase 3 : \textit{Physical Crime Scene} (Scène de crime physique)
        \item Phase 4 : \textit{Digital Crime Scene} (Scène de crime numérique)
        \item Phase 5 : \textit{Review} (Révision)
    \end{enumerate}

    \item \textbf{Le Modèle ISO/IEC 27037:2012} : Ce standard international fournit des lignes directrices pour :
    \begin{itemize}[leftmargin=*]
        \item L'Identification des preuves numériques.
        \item La Collection/Acquisition des preuves.
        \item La Préservation des preuves.
        \item La Documentation des processus.
    \end{itemize}
\end{itemize}
\subsection{6.1 Normes et Standards Internationaux}
\begin{itemize}[leftmargin=*]
    \item \textbf{ISO/IEC 27037:2012} : ``Technologies de l'information — Techniques de sécurité — Lignes directrices pour l'identification, la collecte, l'acquisition et la conservation des preuves numériques''.
    \item \textbf{ISO/IEC 27041:2015} : ``Lignes directrices pour assurer la pertinence et l'adéquation des méthodes d'enquête sur les incidents''.
    \item \textbf{ISO/IEC 27042:2015} : ``Lignes directrices pour l'analyse et l'interprétation des preuves numériques''.
    \item \textbf{ISO/IEC 27043:2015} : ``Principes et processus d'enquête sur les incidents''.
    \item \textbf{NIST SP 800-86} : ``Guide pour l'intégration des techniques forensiques dans la réponse aux incidents''.
    \item \textbf{RFC 3227 (BCP 55)} : ``Lignes directrices pour la collecte et l'archivage des preuves''.
\end{itemize}

L'investigation numérique s'adapte différemment selon les contextes géopolitiques, juridiques et culturels, présentant cette diversité comme une richesse méthodologique et un défi d'harmonisation. Les cas mondiaux sont souvent évalués selon le framework du \textbf{Trilemme CRO (Confidentialité, Fiabilité, Opposabilité juridique)}.


\section{Cadre Normatif et Réglementaire}
\begin{itemize}[leftmargin=*]
    \item ISO/IEC 27037, ISO/IEC 27041-27043
    \item NIST SP 800-86
    \item RFC 3227
    \item ACPO Good Practice Guide
    \item Adaptation aux environnements Cloud et IoT : ISO/IEC 27050, CSA Guidelines, IEEE P2933, ETSI TR 103 939
\end{itemize}

\section{Applications et Cas d’Usage}
\subsection{Local – Cameroun}

Le Cameroun sert d'exemple pour illustrer l'application de l'investigation numérique dans un contexte local.

\subsubsection{7.1.1 Environnement d’Entreprise : Fuite de Données Sensibles}
\begin{itemize}[leftmargin=*]
    \item \textbf{Contexte et Incident} : Une entreprise pharmaceutique de 10 000 employés subit une fuite de formules brevetées.
    \item \textbf{Méthodologie appliquée (ISO 27043)} :
    \begin{itemize}[leftmargin=*]
        \item \textit{Détection} : Alerte d'un système de prévention des pertes de données (DLP).
        \item \textit{Préservation} : Création de snapshots de machines virtuelles suspectes, isolation réseau et préservation des logs via un SIEM.
        \item \textit{Collecte} : Acquisition d'images disque des postes de travail (avec dd, dcfldd), exportation des logs centralisés et capture du trafic réseau (tcpdump).
        \item \textit{Analyse} : Utilisation d'outils comme Plaso/log2timeline pour l'analyse chronologique, RegRipper pour l'analyse du registre et examen des fichiers PST pour les e-mails.
        \item \textit{Résultats} : Identification d'une menace interne (insider threat), reconstruction du chemin d'exfiltration et création d'un paquet de preuves.
    \end{itemize}
\end{itemize}

\subsection{International}
Exemples : cyber-espionnage industriel, ransomwares, manipulation électorale, cyberterrorisme, fraude mobile, criminalité environnementale digitale, narcotrafic numérique. Chaque contexte nécessite une adaptation méthodologique selon le Trilemme CRO (Confidentialité, Fiabilité, Opposabilité juridique).

L’investigation numérique est une discipline dynamique, intégrant intelligence artificielle, big data et cryptographie post-quantique. Elle repose sur l’éthique, la rigueur scientifique et la responsabilité sociale. L’excellence d’un investigateur se mesure à son engagement moral et sa capacité à protéger la vérité, la justice et la mémoire collective dans un monde numérique complexe et globalisé.

\end{document}
