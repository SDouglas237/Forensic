\documentclass[12pt]{article}
\usepackage[a4paper,margin=2.5cm]{geometry}
\usepackage{tikz}
\usepackage{graphicx}
\usepackage{url}
\usepackage{pdfpages}
\usepackage{xcolor}
\usepackage{multicol}
\usepackage{ragged2e}
\usetikzlibrary{calc}
\usepackage[utf8]{inputenc} 
\usepackage[T1]{fontenc}   
\usepackage{lmodern}        
\usepackage{geometry}
\usepackage{enumitem}       
\usepackage{babel}
\usepackage{lmodern}
\usepackage{babel}
\usepackage{amsmath}
\usepackage{array} 
\usepackage{booktabs}
\usepackage{hyperref}       

\begin{document}
\thispagestyle{empty}

% --- Bordure sur toute la page ---
\begin{tikzpicture}[remember picture, overlay]
  \draw[line width=4pt, orange]
    ($(current page.north west) + (1cm,-1cm)$)
    rectangle
    ($(current page.south east) + (-1cm,1cm)$);
\end{tikzpicture}

\vspace*{1.5cm}

% --- Contenu principal : trois colonnes (gauche, centre, droite) ---
\noindent
\begin{minipage}{0.3\linewidth}
    \begin{center}
        \textbf{RÉPUBLIQUE DU CAMEROUN}\\
        *****\\
        Paix - Travail - Patrie\\
        *****\\[0.3cm]

        \textbf{UNIVERSITÉ DE YAOUNDÉ I}\\
        *****\\[0.3cm]

        ÉCOLE NATIONALE SUPÉRIEURE\\
        POLYTECHNIQUE DE YAOUNDÉ\\
        *****\\[0.3cm]

        \textbf{DÉPARTEMENT DE GÉNIE}\\
        ***** 
    \end{center}
\end{minipage}
\begin{minipage}{0.35\linewidth}
    \begin{center}
        % Logos au centre
        \includegraphics[width=3cm]{uy1.png}\\[0.5cm]
        \includegraphics[width=3cm]{polytech.png}
    \end{center}
\end{minipage}
\begin{minipage}{0.3\linewidth}
    \begin{center}
        \textbf{REPUBLIC OF CAMEROON}\\
        *****\\
        Peace - Work - Fatherland\\
        *****\\[0.3cm]

        \textbf{UNIVERSITY OF YAOUNDE I}\\
        *****\\[0.3cm]

        NATIONAL ADVANCED SCHOOL\\
        OF ENGINEERING OF YAOUNDE\\
        *****\\[0.3cm]

        \textbf{COMPUTER ENGINEERING}\\
        ***** 
    \end{center}
\end{minipage}

\vspace{2cm}

% --- Sujet au centre ---
\begin{center} 
	\colorbox{orange!80}{
        \parbox{\dimexpr\textwidth-4\fboxsep}{
            \centering
            {\LARGE \textbf{Techniques et pratiques de l'investigation numérique}}
        }
    }\\[0.8cm]
    {\large \textbf{Sous le thème :} Résumé Techniques et pratiques de l'investigation numérique}\\[1.5cm]

    \textbf{{EMBOLO MVOGO SHAWN DOUGLAS}}\\[0.8cm]
    \textbf{{Matricule : 22P072}}\\[0.8cm]
    \textbf{{Filière : CIN4}}\\[0.8cm]
    Sous la supervision de : \textbf{Mr Minka}\\[1.5cm]

    Année Scolaire : \textbf{2025 -- 2026}
\end{center}


\newpage
\setcounter{page}{1}

\newpage

\begin{center}
\Large\textbf{RAPPORT ACADÉMIQUE SUR JOSÉ LOÏC MIYANDA YUNGUI}\\[0.4cm]
\large Étudiant ingénieur en cybersécurité à l’École Nationale Supérieure Polytechnique de Yaoundé
\end{center}

\hrule
\vspace{0.5cm}

\section*{Résumé}
Ce rapport présente un portrait académique et professionnel détaillé de \textbf{José Loïc Miyanda Yungui}, connu sous le pseudonyme \textbf{Jylo}. Étudiant ingénieur à l’École Nationale Supérieure Polytechnique de Yaoundé (ENSPY), il se distingue par son engagement dans la cybersécurité, la transformation digitale et la création numérique.  
L’objectif de ce rapport est d’analyser son parcours académique, ses expériences professionnelles et associatives, ainsi que sa présence numérique, dans un cadre académique rigoureux.

\vspace{0.5cm}
\hrule
\vspace{0.5cm}

\section*{Identité et informations générales}

\begin{center}
\begin{tabular}{|p{6cm}|p{8cm}|}
\hline
\textbf{Élément} & \textbf{Information} \\
\hline
Nom complet & José Loïc Miyanda Yungui \\
\hline
Surnoms / Pseudonymes & Jylo, Le Lion \\
\hline
Nationalité & Camerounaise \\
\hline
Adresse e-mail professionnelle & \texttt{jylolelion237@gmail.com} \\
\hline
Ville de résidence & Yaoundé, Cameroun \\
\hline
Statut & Étudiant ingénieur en Cybersécurité \\
\hline
Établissement actuel & École Nationale Supérieure Polytechnique de Yaoundé (ENSPY) \\
\hline
Niveau & 4\textsuperscript{e} année de spécialité CIN \\
\hline
Rôle associatif & Fondateur et Manager de Polycœur Org \\
\hline
Autres activités & Digital Creator, ancien Directeur Commercial \& Marketing à l’Académie du Codeur \\
\hline
\end{tabular}
\end{center}

\vspace{0.5cm}
\hrule
\vspace{0.5cm}

\section*{Parcours académique}

\subsection*{1. Études secondaires}
José Loïc Miyanda Yungui a effectué son parcours scolaire au \textbf{Collège Marie Albert 2} avant de poursuivre au \textbf{Lycée Dahala}, où il a obtenu son \textbf{Baccalauréat scientifique} en 2019.

\subsection*{2. Études universitaires}
Après le baccalauréat, il s’inscrit à la \textbf{Faculté des Sciences de l’Université de Yaoundé I}, dans la filière \textbf{Biosciences}, où il effectue deux années d’études.

\subsection*{3. Formation d’ingénieur}
Intégré à l’\textbf{École Nationale Supérieure Polytechnique de Yaoundé}, il rejoint la \textbf{filière Humanité Numérique} avant d’entrer en spécialité \textbf{Cybersécurité et Investigation Numérique (CIN)}.  
Il poursuit actuellement sa \textbf{quatrième année de spécialité}.  
Sa formation s’articule autour de la sécurité informatique, du développement de solutions numériques et de la gestion de projets de digitalisation.

\vspace{0.5cm}
\hrule
\vspace{0.5cm}

\section*{Expérience professionnelle}

\subsection*{1. L’Académie du Codeur}
José Loïc Miyanda Yungui a occupé le poste de \textbf{Directeur Commercial et Marketing} à l’Académie du Codeur, structure camerounaise dédiée à la formation numérique et à l’employabilité des jeunes.  
Il y a contribué à la mise en place de stratégies de communication digitale, au développement de partenariats éducatifs et à la valorisation des compétences technologiques locales.

\subsection*{2. Création de contenu digital}
Sous le pseudonyme \textbf{Jylo}, il se positionne également comme \textbf{Digital Creator}, produisant des contenus éducatifs, humoristiques et motivants sur les réseaux sociaux.

\vspace{0.5cm}
\hrule
\vspace{0.5cm}

\section*{Engagement associatif}

\subsection*{1. Polycœur Org}
Fondateur et manager de \textbf{Polycœur Org}, José Loïc Miyanda Yungui s’investit dans des initiatives sociales et éducatives au sein de l’ENSPY.  
L’association promeut la solidarité étudiante, la sensibilisation à la santé mentale et la participation communautaire au sein du campus.

\subsection*{2. Leadership étudiant}
À travers ses activités, il démontre un leadership affirmé et un esprit de collaboration.  
Il incarne le modèle d’un ingénieur citoyen, conscient de son rôle social et du pouvoir du numérique pour l’inclusion.

\vspace{0.5cm}
\hrule
\vspace{0.5cm}

\section*{Présence numérique et communication}

\begin{center}
\begin{tabular}{|p{3cm}|p{8cm}|p{4cm}|}
\hline
\textbf{Plateforme} & \textbf{Lien / Nom} & \textbf{Statut / Observations} \\
\hline
TikTok & \href{https://vm.tiktok.com/ZMHvYLAsUXAqn-XRm8Y/}{@jylo.kmerstarbatle} & Compte personnel – création de contenu digital \\
\hline
Facebook (1) & \href{https://www.facebook.com/profile.php?id=61558227841643}{Profil personnel (2025)} & Compte actif, plus récent \\
\hline
Facebook (2) & \href{https://www.facebook.com/profile.php?id=61575315117309}{Profil personnel secondaire} & Publications mixtes (professionnelles et sociales) \\
\hline
Facebook (3) & \href{https://www.facebook.com/jlolekamer}{Page publique Jylo le Kamer} & Page orientée communication digitale \\
\hline
Facebook (4) & \href{https://www.facebook.com/jylo.minbejo}{Page Jylo Minbejo} & Ancien compte ou secondaire \\
\hline
LinkedIn & José Loïc Minyanda (x2 profils) & Comptes professionnels liés à la cybersécurité et à Polycœur Org \\
\hline
\end{tabular}
\end{center}

\vspace{0.5cm}
Sa présence numérique témoigne d’une stratégie d’identité plurielle : à la fois ingénieur, créateur et communicateur.  
Il représente une génération d’ingénieurs connectés, maîtrisant les codes du numérique autant que ceux de la technique.

\vspace{0.5cm}
\hrule
\vspace{0.5cm}

\section*{Centres d’intérêt}
\begin{itemize}
\item Informatique
\item Développement personnel
\item Musique et art digital
\item Entrepreneuriat
\end{itemize}
\end{document}
