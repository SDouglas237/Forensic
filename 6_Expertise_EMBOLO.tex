\documentclass[12pt]{article}
\usepackage[a4paper,margin=2.5cm]{geometry}
\usepackage{tikz}
\usepackage{graphicx}
\usepackage{url}
\usepackage{pdfpages}
\usepackage{xcolor}
\usepackage{multicol}
\usepackage{ragged2e}
\usetikzlibrary{calc}
\usepackage[utf8]{inputenc} 
\usepackage[T1]{fontenc}   
\usepackage{lmodern}        
\usepackage{geometry}
\usepackage{enumitem}       
\usepackage{babel}
\usepackage{lmodern}
\usepackage{babel}
\usepackage{amsmath}
\usepackage{array} 
\usepackage{booktabs}
\usepackage{hyperref}       

\begin{document}
\thispagestyle{empty}

% --- Bordure sur toute la page ---
\begin{tikzpicture}[remember picture, overlay]
  \draw[line width=4pt, orange]
    ($(current page.north west) + (1cm,-1cm)$)
    rectangle
    ($(current page.south east) + (-1cm,1cm)$);
\end{tikzpicture}

\vspace*{1.5cm}

% --- Contenu principal : trois colonnes (gauche, centre, droite) ---
\noindent
\begin{minipage}{0.3\linewidth}
    \begin{center}
        \textbf{RÉPUBLIQUE DU CAMEROUN}\\
        *****\\
        Paix - Travail - Patrie\\
        *****\\[0.3cm]

        \textbf{UNIVERSITÉ DE YAOUNDÉ I}\\
        *****\\[0.3cm]

        ÉCOLE NATIONALE SUPÉRIEURE\\
        POLYTECHNIQUE DE YAOUNDÉ\\
        *****\\[0.3cm]

        \textbf{DÉPARTEMENT DE GÉNIE}\\
        ***** 
    \end{center}
\end{minipage}
\begin{minipage}{0.35\linewidth}
    \begin{center}
        % Logos au centre
        \includegraphics[width=3cm]{uy1.png}\\[0.5cm]
        \includegraphics[width=3cm]{polytech.png}
    \end{center}
\end{minipage}
\begin{minipage}{0.3\linewidth}
    \begin{center}
        \textbf{REPUBLIC OF CAMEROON}\\
        *****\\
        Peace - Work - Fatherland\\
        *****\\[0.3cm]

        \textbf{UNIVERSITY OF YAOUNDE I}\\
        *****\\[0.3cm]

        NATIONAL ADVANCED SCHOOL\\
        OF ENGINEERING OF YAOUNDE\\
        *****\\[0.3cm]

        \textbf{COMPUTER ENGINEERING}\\
        ***** 
    \end{center}
\end{minipage}

\vspace{2cm}

% --- Sujet au centre ---
\begin{center} 
	\colorbox{orange!80}{
        \parbox{\dimexpr\textwidth-4\fboxsep}{
            \centering
            {\LARGE \textbf{Techniques et pratiques de l'investigation numérique}}
        }
    }\\[0.8cm]
    {\large \textbf{Sous le thème :} Résumé Techniques et pratiques de l'investigation numérique}\\[1.5cm]

    \textbf{{EMBOLO MVOGO SHAWN DOUGLAS}}\\[0.8cm]
    \textbf{{Matricule : 22P072}}\\[0.8cm]
    \textbf{{Filière : CIN4}}\\[0.8cm]
    Sous la supervision de : \textbf{Mr Minka}\\[1.5cm]

    Année Scolaire : \textbf{2025 -- 2026}
\end{center}


\newpage
\setcounter{page}{1}

\newpage


\section*{Différents éléments constituant l’investigation numérique}

\subsection*{1. Tableau organisé des éléments d’enquête}

\begin{enumerate}[label=\textbf{E\arabic* —}, leftmargin=2cm]

\item \textbf{Message WhatsApp reçu par la victime} \\
\textbf{Description :} message texte reçu par la victime, potentiellement incriminant. \\
\textbf{Source :} téléphone de la victime (application WhatsApp).

\item \textbf{Appel téléphonique reçu (signalé par la victime)} \\
\textbf{Description :} appel entrant pouvant relier l’appelant à l’évènement. \\
\textbf{Source :} opérateur télécom, téléphone de la victime (logs d’appels).

\item \textbf{Données de localisation téléphonique (géolocalisation)} \\
\textbf{Description :} positions permettant d’attester de la présence géographique des acteurs. \\
\textbf{Source :} opérateurs mobiles, téléphone (historique GPS), applications (WhatsApp – partage de position).

\item \textbf{Listing d’appels (historique d’appels)} \\
\textbf{Description :} liste d’appels entrants/sortants (dates, heures, durées), utile pour confirmer les appels et corréler les événements. \\
\textbf{Source :} opérateur télécom, téléphone du suspect ou de la victime.

\item \textbf{Messages WhatsApp liés à la fiche technique de localisation} \\
\textbf{Description :} messages contenant coordonnées, captures de position, instructions logistiques. \\
\textbf{Source :} conversation WhatsApp (émetteur/récepteur).

\item \textbf{Visionnage / vidéosurveillance du bureau} \\
\textbf{Description :} enregistrements vidéo montrant les mouvements ou actes au bureau (date, heure, personnes, véhicule). \\
\textbf{Source :} caméras de surveillance.

\item \textbf{Document prouvant la véritable identité du prétendu « capitaine »} \\
\textbf{Description :} pièces d’identité, certificats ou documents officiels, falsifiables ou non. Permet d’identifier la personne réelle derrière le pseudonyme. \\
\textbf{Source :} documents papier, copies numériques, bases administratives.

\item \textbf{Bande sonore recueillie sur le téléphone (enregistrement)} \\
\textbf{Description :} enregistrement audio sur mobile pouvant attester d’une conversation incriminante. \\
\textbf{Source :} fichier audio sur téléphone du témoin, de l’accusé ou de la victime.

\item \textbf{Images de vidéosurveillance d’un lieu public (ville)} \\
\textbf{Description :} vidéos montrant déplacements, véhicules, plaques d’immatriculation, etc. \\
\textbf{Source :} caméras publiques.

\item \textbf{Exploitation du listing d’appels pour confirmer appel et envoi de deux SMS} \\
\textbf{Description :} preuves de communication texte et voix corroborant présence ou activité. \\
\textbf{Source :} opérateur mobile, téléphone.

\item \textbf{Données de géolocalisation concordantes entre plusieurs sources} \\
\textbf{Description :} convergence des traces GPS renforçant la crédibilité du positionnement. \\
\textbf{Source :} opérateur, téléphones, applications.

\item \textbf{Thermocopie / capture d’écran des historiques d’appels du téléphone du défunt} \\
\textbf{Description :} captures d’écran extraites et imprimées montrant l’historique d’appels. \\
\textbf{Source :} téléphone du défunt.

\item \textbf{Copie des historiques des transactions bancaires (financement des opérations)} \\
\textbf{Description :} relevés bancaires montrant virements et paiements liés au financement des opérations incriminées. \\
\textbf{Source :} banques (relevés électroniques), comptes bancaires des suspects.

\item \textbf{Reçus de paiement de la location du véhicule utilisé pour le délit} \\
\textbf{Description :} factures, reçus ou contrats de location liant le véhicule aux suspects. \\
\textbf{Source :} société de location.

\end{enumerate}

\end{document}
