\documentclass[12pt]{article}
\usepackage{graphicx}
\usepackage{geometry}
\usepackage[utf8]{inputenc}
\usepackage[T1]{fontenc}
\usepackage{ulem}
\usepackage{enumitem}
\geometry{margin=1in}
\usepackage[utf8]{inputenc}
\usepackage[T1]{fontenc}
\usepackage[french]{babel}
\usepackage{setspace}
\usepackage{csquotes}
\usepackage{hyperref}
\setstretch{1.2}
\usepackage[utf8]{inputenc}
\usepackage[T1]{fontenc}
\usepackage[french]{babel}
\usepackage{titlesec}
\usepackage{amsmath}
\usepackage{amssymb}

\geometry{a4paper, margin=2.5cm}
\setlength{\parindent}{0pt}
\setlength{\parskip}{1em}

\titleformat{\section}{\Large\bfseries}{}{0em}{}
\titleformat{\subsection}{\large\bfseries}{}{0em}{}

\begin{document}

\thispagestyle{empty}

% --- Trois colonnes symétriques ---
\begin{minipage}{0.3\linewidth}
    \begin{center}
        \textbf{RÉPUBLIQUE DU CAMEROUN}\\
        *****\\
        Paix - Travail - Patrie\\
        *****\\[0.3cm]

        \textbf{UNIVERSITÉ DE YAOUNDÉ I}\\
        *****\\[0.3cm]

        ÉCOLE NATIONALE SUPÉRIEURE\\
        POLYTECHNIQUE DE YAOUNDÉ\\
        *****\\[0.3cm]

        \textbf{DÉPARTEMENT DE GÉNIE}\\
        ***** 
    \end{center}
\end{minipage}
\begin{minipage}{0.35\linewidth}
    \begin{center}
        % Logos au centre
        \includegraphics[width=3cm]{uy1.png}\\[0.5cm]
        \includegraphics[width=3cm]{polytech.png}
    \end{center}
\end{minipage}
\begin{minipage}{0.3\linewidth}
    \begin{center}
        \textbf{REPUBLIC OF CAMEROON}\\
        *****\\
        Peace - Work - Fatherland\\
        *****\\[0.3cm]

        \textbf{UNIVERSITY OF YAOUNDE I}\\
        *****\\[0.3cm]

        NATIONAL ADVANCED SCHOOL\\
        OF ENGINEERING OF YAOUNDE\\
        *****\\[0.3cm]

        \textbf{COMPUTER ENGINEERING}\\
        ***** 
    \end{center}
\end{minipage}

\vspace{2cm}

% --- Sujet au centre ---
\begin{center}
    {\LARGE \textbf{Techniques et pratiques de l'investigation numérique}}\\[0.5cm]
    {\large \textbf{Sous le thème :}Resumé Techniques et pratiques de l'investigation numérique}\\[2cm]

    \textbf{{EMBOLO MVOGO SHAWN DOUGLAS}}\\[0.8cm]
    \textbf{{22P072}}\\[0.8cm]
    \textbf{{Filiére : CIN4}}\\[0.8cm]
    Sous la supervision de : \textbf{Mr Minka}\\[0.8cm]
    Année Scolaire : \textbf{2025 -- 2026}
\end{center}
\section*{Partie 1 : Fondements Philosophiques et Épistémologiques}

\subsection*{1. Analyse Critique du Paradoxe de la Transparence}

Pour Han, notre société valorise la transparence absolue : dans la politique, l'économie, la communication et surtout les relations interpersonnelles. Elle est perçue comme synonyme de démocratie, de vérité et de liberté.

Mais le paradoxe est le suivant :

\begin{itemize}
\item Plus on exige de transparence, moins il y a de confiance, de liberté et d'intimité.
\item La transparence, censée libérer, produit au contraire une nouvelle forme de contrôle et de domination.
\item Elle transforme les citoyens en objets de surveillance, et les relations humaines en échanges marchands, mesurables et calculables.
\end{itemize}

En d'autres termes, vouloir tout rendre visible détruit la possibilité du mystère, de la distance, du secret et donc de la liberté véritable.

Byung-Chul Han, philosophe contemporain, a marqué la pensée critique du XXI\textsuperscript{e} siècle par son analyse de la société numérique et néolibérale. Dans \emph{La société de la transparence}, il met en évidence un paradoxe troublant : alors que la transparence est érigée comme une valeur suprême, garante de liberté et de vérité, elle engendre en réalité une nouvelle forme d'aliénation. Loin d'émanciper l'individu, l'obsession de la visibilité totale produit un contrôle diffus, une perte de confiance et une marchandisation des rapports humains.

Tout d'abord, la transparence se présente comme un idéal démocratique. Dans l'imaginaire collectif, rendre tout visible et accessible permettrait de lutter contre la corruption, les mensonges et les manipulations. La transparence est donc associée à la sincérité, à l'égalité et à la responsabilité. À première vue, elle semble constituer un progrès : plus rien n'échappe au regard collectif, chacun peut vérifier par lui-même, et le pouvoir paraît réduit à une exposition permanente. En ce sens, la transparence se présente comme le fondement d'une société ouverte.

Cependant, Han révèle le caractère illusoire de cet idéal. La transparence repose sur une logique de méfiance : si l'on exige que tout soit visible, c'est parce que l'on ne fait plus confiance. Or, une société qui abolit le secret, l'opacité et l'intimité ne crée pas plus de liberté, mais une surveillance permanente. Les individus deviennent des données à collecter, à analyser et à exploiter. Le paradoxe est alors évident : la transparence, censée garantir la confiance, détruit les conditions mêmes de la confiance, puisqu'elle repose sur un contrôle constant.

De plus, la transparence transforme la communication et les relations humaines. Sur les réseaux sociaux, chacun se met en scène, persuadé de s'exprimer librement. Mais cette visibilité volontaire s'inscrit dans une logique marchande : l'individu devient un produit, évalué en « likes », en partages ou en influence. Ce qui se voulait expression authentique se transforme en performance calibrée. Le paradoxe apparaît encore : l'hypervisibilité, censée renforcer l'identité, aboutit à sa dissolution dans une logique de consommation et de comparaison permanente.

Enfin, la transparence appauvrit l'expérience humaine en abolissant le secret et le mystère. Pour Han, l'opacité, le non-dit et la distance sont essentiels à la liberté, à l'art et même à l'amour. L'obsession de la clarté totale réduit les relations à des transactions et détruit la profondeur. Paradoxalement, plus tout est visible, plus tout devient uniforme et superficiel. La lumière totale, au lieu d'éclairer, éblouit et empêche de voir.

En conclusion, le paradoxe de la transparence, tel que l'analyse Byung-Chul Han, réside dans ce renversement : ce qui est présenté comme une conquête démocratique et un instrument de liberté se révèle être un nouveau mode de domination. La transparence absolue, loin de libérer, engendre surveillance, marchandisation et superficialité. Elle transforme la société en un espace sans secret, où la confiance disparaît et où l'homme perd une part essentielle de sa liberté intérieure. Penser ce paradoxe, c'est donc rappeler la valeur irréductible de l'opacité, du silence et du mystère dans toute vie authentiquement humaine.

\textbf{Source :} « La société de la transparence » de Byung-Chul Han - El Correo \\
Byung-Chul Han, LA SOCIÉTÉ DE TRANSPARENCE | Cairn.info

\subsubsection*{Application du Paradoxe de la Transparence : Le Cas des Données de Santé Publique}

Un cas concret illustrant parfaitement le paradoxe de la transparence se trouve dans la gestion des données de santé pendant une crise sanitaire. Prenons l'exemple de la publication de données COVID-19 par les gouvernements.

\paragraph{Le Paradoxe en Action :}
D'un côté, la transparence exigeait la publication détaillée des statistiques (taux d'infection par quartier, lieux de contamination, données démographiques). Cette transparence était essentielle pour :
\begin{itemize}
\item Informer le public
\item Justifier les mesures sanitaires
\item Permettre la recherche scientifique
\end{itemize}

Mais d'un autre côté, cette transparence créait :
\begin{itemize}
\item Une stigmatisation des quartiers à forte incidence
\item Des risques de réidentification des patients
\item Une surveillance potentielle des déplacements individuels
\item Une pression sociale sur les personnes infectées
\end{itemize}

\paragraph{La Résolution Kantienne :}
L'impératif catégorique de Kant offre une voie médiane. Selon sa première formulation : « Agis uniquement d'après la maxime grâce à laquelle tu peux vouloir en même temps qu'elle devienne une loi universelle. »

Appliqué à notre cas :

\begin{itemize}
\item \textbf{Universalisation du Principe :} La maxime devrait être : « Je divulguerai des données de santé uniquement si ce principe peut être universalisé sans contradiction. »
\item \textbf{Respect de l'Humanité :} La seconde formulation kantienne -- « Agis de telle sorte que tu traites l'humanité toujours en même temps comme une fin, et jamais simplement comme un moyen » -- exige que les données des citoyens ne soient pas simplement utilisées comme moyens pour une fin collective, mais que leur dignité soit préservée.
\end{itemize}

\paragraph{Application Pratique :}

\begin{itemize}
\item \textbf{Transparence Différenciée :}
  \begin{itemize}
  \item Données agrégées au niveau communal plutôt qu'individuelles
  \item Délai de publication permettant l'anonymisation
  \item Niveaux de granularité adaptés aux différents publics
  \end{itemize}
  
\item \textbf{Cadre Éthique Préalable :}
  \begin{itemize}
  \item Tout système de collecte doit intégrer des garde-fous éthiques avant sa mise en œuvre
  \item Principe de proportionnalité : la transparence doit être proportionnée au besoin social
  \end{itemize}
  
\item \textbf{Consentement Éclairé :}
  \begin{itemize}
  \item Information claire sur l'usage des données
  \item Possibilité de retrait pour les données non essentielles
  \end{itemize}
  
\item \textbf{Supervision Indépendante :}
  \begin{itemize}
  \item Comité d'éthique permanent pour évaluer les demandes d'accès
  \item Audit régulier des systèmes de protection
  \end{itemize}
\end{itemize}

Cette approche kantienne permet de dépasser le faux dilemme entre transparence absolue et opacité totale. Elle reconnaît que la transparence est un moyen au service de la dignité humaine, non une fin en soi. Le gouvernement devient ainsi non pas un simple diffuseur de données, mais un garant de leur usage éthique.

\subsection*{2. Transformation Ontologique du Numérique}

\subsubsection*{La Transformation Ontologique du Numérique : De l'Être-heideggérien à l'Être-par-la-trace}

\paragraph{1. L'Être comme "Dasein" chez Heidegger et sa Transformation Numérique}

Chez Heidegger, l'être humain est un « Dasein » (être-là), caractérisé par :
\begin{itemize}
\item Son être-au-monde (In-der-Welt-sein)
\item Sa temporalité
\item Son souci (Sorge)
\item Son authenticité face à l'être-pour-la-mort
\end{itemize}

À l'ère numérique, cette conception subit une mutation fondamentale. Le Dasein devient un « Être-digital » dont l'existence se déploie simultanément dans le monde physique et l'espace numérique. Cette dualité crée une nouvelle condition ontologique où l'être n'est plus seulement incarné mais aussi datifié.

\paragraph{2. Analyse d'un Profil Social comme "Être-par-la-trace"}

Prenons l'exemple d'un profil Instagram complet :

\begin{itemize}
\item \textbf{Trace comme Manifestation d'Existence :}
  \begin{itemize}
  \item Photos géolocalisées : Manifestent l'être-au-monde numérique
  \item Likes et commentaires : Témoignent de l'être-avec-autrui (Mitsein) digital
  \item Historique des stories : Illustrent la temporalité numérique
  \item Abonnements/abonnés : Définissent l'être-en-relation
  \end{itemize}
\end{itemize}

Ce profil constitue une ontologie parallèle où chaque trace numérique devient un mode de manifestation de l'être. Comme le souligne Heidegger dans \emph{Être et Temps}, l'être se révèle dans son être-au-monde -- ici, le monde numérique.

\paragraph{3. L'Impact sur la Notion de Preuve Légale}

\textbf{Transformation Épistémologique :}
\begin{itemize}
\item Preuve testimoniale → Preuve algorithmique
\item Témoignage humain → Trace numérique objective
\item Mémoire subjective → Data immuable
\end{itemize}

\textbf{Nouveaux Paradigmes Juridiques :}

\begin{itemize}
\item \textbf{Preuve comme "Évidence Numérique" :}
  \begin{itemize}
  \item Les métadonnées deviennent des témoins silencieux
  \item L'historique de navigation constitue une « intentionnalité enregistrée »
  \item La géolocalisation crée une « présence prouvée »
  \end{itemize}
\end{itemize}

\textbf{Problématiques :}
\begin{itemize}
\item Authenticité vs Réplication : La facilité de falsification des données
\item Contextualisation : Une trace hors de son contexte ontologique initial
\item Consentement ontologique : L'être est-il conscient de créer des preuves permanentes ?
\end{itemize}

\paragraph{4. Conséquences Ontologiques Profondes}

\begin{itemize}
\item \textbf{Altération de l'Être-pour-la-Mort :} La trace numérique survit à la mort biologique, créant un « être-pour-la-donnée » qui défie la finitude heideggérienne.
\item \textbf{Double Numérique comme Aliénalité :} Le profil social échappe à son origine pour devenir une entité autonome, actualisant la conception heideggérienne de l'aliénation dans l'ère numérique.
\end{itemize}

\paragraph{Conclusion :} Nous assistons à une métamorphose de l'être lui-même, où le Dasein doit désormais composer avec son existence digitale. Cette transformation exige une refondation de nos cadres juridiques et éthiques pour appréhender cette nouvelle manifestation de l'être dans son être-par-la-trace.

\section*{Partie 3 : Révolution Quantique et Ses Implications}

\subsection*{6. Expérience de Pensée Schrödinger Adaptée}

\subsubsection*{— Concevez une version numérique du chat de Schrödinger}

Dans l'expérience originale, un chat est placé dans une boîte avec un dispositif quantique : tant que la boîte n'est pas ouverte, le chat est à la fois « vivant » et « mort », dans un état superposé.

\paragraph{Le Message de Schrödinger}

Situation : Vous recevez une notification : « Nouveau message de Paul ». Mais une note précise : « Ce message s'auto-détruira peut-être à l'ouverture ». 

Avant d'ouvrir le message :
\begin{itemize}
\item Le message existe ET n'existe pas en même temps
\item Il contient peut-être une invitation à dîner
\item Il contient peut-être rien du tout
\item C'est une photo incroyable
\item C'est un truc ennuyeux
\item Ça va changer votre vie
\item C'est sans importance en fait
\end{itemize}

Le message existe dans tous ses états possibles et toutes ces réalités coexistent dans cet instant précis.

Quand vous ouvrez le message :
\begin{itemize}
\item Soit vous voyez le texte (le « chat » est vivant)
\item Soit l'écran devient vide (le « chat » est mort)
\end{itemize}

Le paradoxe : Jusqu'au moment où vous regardez, le message est à la fois « existant » et « inexistant ». C'est votre action de regarder qui décide de son sort.

\subsubsection*{— Un fichier existe-t-il dans un état superposé « présent/effacé » avant analyse?}

On peut répondre en distinguant trois niveaux : physique, logique et juridique.

\paragraph{1. Niveau physique :}
Dans un disque dur, une clé USB ou une mémoire flash, lorsqu'on « efface » un fichier :
\begin{itemize}
\item En réalité, les données binaires sont toujours présentes tant qu'elles n'ont pas été réécrites
\item L'effacement ne fait souvent que marquer l'espace comme « libre » dans la table d'allocation
\end{itemize}

Techniquement, donc, le fichier est dans un état ambigu :
\begin{itemize}
\item Effacé du point de vue du système (OS ne le reconnaît plus comme existant)
\item Présent dans la matière (les bits restent inscrits dans les cellules mémoires)
\end{itemize}

C'est déjà une superposition pratique : le fichier est « absent » pour l'utilisateur, mais « présent » pour l'analyste qui lit directement la mémoire.

\paragraph{2. Niveau logique (épistémique) :}
Avant qu'un enquêteur n'analyse l'appareil, on ne sait pas si :
\begin{itemize}
\item le fichier est encore récupérable (non réécrit) ou
\item définitivement perdu (données écrasées)
\end{itemize}

On peut donc dire que le fichier est dans un état épistémiquement superposé :
\begin{itemize}
\item $|\text{Présent}\rangle$ → encore récupérable dans les blocs mémoire
\item $|\text{Effacé}\rangle$ → irrémédiablement détruit
\end{itemize}

L'« observation » (analyse forensic) agit comme la mesure quantique : elle fait « s'effondrer » l'incertitude et révèle l'un des deux états.

\paragraph{3. Niveau juridique}

En droit de la preuve :
\begin{itemize}
\item Un fichier « effacé » mais techniquement récupérable peut valoir preuve s'il est extrait correctement et validé (chaîne de custody)
\item Mais s'il a été réécrit, alors il n'existe plus légalement : aucune trace exploitable
\end{itemize}

Ainsi, la « superposition » a un impact crucial : avant expertise, on ne peut affirmer avec certitude si la preuve existe encore.

\subsubsection*{— Quel impact sur la notion de preuve « certaine » en justice?}

Si on applique cette idée du « fichier dans un état superposé présent/effacé » à la justice, l'impact est majeur sur la notion de preuve certaine.

\paragraph{1. Fragilité de la certitude :}
En droit, une preuve certaine doit établir un fait de manière claire, sans ambiguïté.

Or, dans le cas numérique :
\begin{itemize}
\item Un fichier effacé peut être récupérable ou non → incertitude \textbf{avant expertise}
\item Même après récupération, des doutes subsistent :
  \begin{itemize}
  \item L'intégrité est-elle garantie ?
  \item L'origine est-elle authentique ?
  \item La donnée n'a-t-elle pas été altérée par l'outil d'extraction ?
  \end{itemize}
\end{itemize}

Ainsi, contrairement à une empreinte physique (ex : ADN), la preuve numérique est instable et dépendante du contexte technique.

\paragraph{2. Déplacement de la certitude :}
La « certitude » n'est plus liée au contenu du fichier en lui-même, mais :
\begin{itemize}
\item à la méthodologie d'extraction (forensic rigoureux, outils validés, chaîne de custody)
\item à la traçabilité (qui garantit qu'on n'a pas altéré la donnée en la mesurant)
\item à la corroboration par d'autres preuves (témoignages, logs serveur, métadonnées externes)
\end{itemize}

La justice doit donc accepter une preuve numérique comme « certaine » seulement si elle est inscrite dans un réseau de garanties procédurales.

\paragraph{3. Transformation de la notion de preuve}

\begin{itemize}
\item \textbf{Avant l'ère numérique :} une preuve était « certaine » car stable (document papier, objet matériel)
\item \textbf{À l'ère numérique :} la preuve est « certaine » si l'incertitude technique est réduite au minimum acceptable par des protocoles normalisés (ex. normes ISO/IEC 27037 en criminalistique numérique)
\end{itemize}

La certitude n'est donc plus absolue, mais procédurale : elle repose sur la confiance dans la méthode d'analyse et non dans l'objet lui-même.

\paragraph{4. Conséquences pratiques en justice :}
\begin{itemize}
\item Charge accrue pour l'expert : il doit démontrer la rigueur et la neutralité de son analyse
\item Possibilité de contestation : un avocat peut toujours soulever le doute sur la récupération ou l'intégrité
\item Nécessité de pluralité des preuves : un seul fichier récupéré peut ne pas suffire ; il faut souvent recouper avec d'autres sources (logs, métadonnées, témoignages)
\end{itemize}

\paragraph{Conclusion :}
Le paradoxe du fichier « présent/effacé » montre que dans le numérique, une preuve « certaine » au sens traditionnel n'existe presque jamais. La certitude judiciaire devient relative, conditionnelle et procédurale : elle dépend de la méthodologie d'investigation, de la traçabilité de la chaîne de custody, et du recoupement avec d'autres preuves.

\subsubsection*{— Rédigez un protocole d'observation minimisant l'effet sur le système}

\paragraph{Protocole d'observation forensique minimalement intrusif}

\begin{enumerate}
\item \textbf{Autorisation légale préalable :}
  \begin{itemize}
  \item Vérifier le mandat ou l'autorisation d'accès
  \item Documenter les limites de l'intervention (périmètre légal)
  \end{itemize}

\item \textbf{Isolation immédiate du support :}
  \begin{itemize}
  \item Déconnecter l'appareil du réseau (mode avion, cage de Faraday) pour éviter toute altération par synchronisation automatique ou effacement à distance
  \item Ne pas interagir avec l'OS (pas de clics, pas d'ouverture d'applications)
  \end{itemize}

\item \textbf{Documentation initiale (sans ouverture) :}
  \begin{itemize}
  \item Photographier l'appareil dans son état initial (écran allumé, notifications, date/heure affichées)
  \item Noter les conditions de saisie (lieu, heure, intervenants)
  \end{itemize}

\item \textbf{Acquisition en lecture seule :}
  \begin{itemize}
  \item Utiliser un write blocker matériel ou logiciel pour éviter toute écriture accidentelle
  \item Réaliser une image forensique complète (dump physique ou logique) du support
  \item Calculer des empreintes cryptographiques (hash MD5/SHA256) immédiatement pour garantir l'intégrité
  \end{itemize}

\item \textbf{Conservation de l'original :}
  \begin{itemize}
  \item Stocker le support saisi scellé, sans jamais y revenir pour l'analyse
  \item Travailler uniquement sur la copie forensique
  \end{itemize}

\item \textbf{Analyse contrôlée :}
  \begin{itemize}
  \item Ouvrir la copie dans un environnement isolé (machine virtuelle, sandbox forensic)
  \item Employer des outils validés (ex. EnCase, Autopsy, FTK, X-Ways) et consigner leurs versions
  \item Ne pas « exécuter » le fichier suspect : analyser ses métadonnées, fragments binaires, signatures
  \end{itemize}

\item \textbf{Corrélation externe :}
  \begin{itemize}
  \item Vérifier l'existence de traces dans les sauvegardes cloud, journaux systèmes, métadonnées réseau
  \item Recouper avec d'autres sources indépendantes pour renforcer la certitude
  \end{itemize}

\item \textbf{Traçabilité et rapport :}
  \begin{itemize}
  \item Documenter chaque action (qui, quoi, quand, comment)
  \item Garantir la chaîne de custody (transferts et manipulations tracés)
  \item Produire un rapport clair, permettant à un tiers de reproduire l'observation
  \end{itemize}
\end{enumerate}

Le protocole repose sur deux principes clés :
\begin{itemize}
\item Ne jamais observer directement sur l'original (comme ouvrir un message view-once)
\item Travailler sur une copie intégrale en lecture seule, afin que l'acte d'observation ne modifie pas l'état du système
\end{itemize}

« La meilleure preuve est celle qu'on observe sans y toucher. »

\end{document}
